\documentclass[compressed,dvips,letter]{beamer}

\mode<presentation>
{
  \usetheme{Boadilla}
  \setbeamercovered{dynamic}
}

%****************************************************************************
%
%****************************************************************************
%
\usepackage{graphicx}
% \usepackage{ifthen}
\usepackage{fancybox}
\usepackage{textpos}
% \usepackage{color}
% \usepackage{hyphenat}
% \renewcommand{\sfdefault}{pag}  % Use a nicer font for the headers
% \usepackage{float}
% \usepackage{paralist}
\usepackage{url}
% \usepackage{moreverb}
% \usepackage{longtable}

\usepackage{bm,bbm}
\usepackage{amsfonts,amsmath,amsthm,amstext,amssymb}
\usepackage{latexsym}
\usepackage[usenames]{pstricks}
%\usepackage{pst-plot,pst-3dplot,pst-node}
\usepackage{psfrag}
\usepackage{enumerate}
\usepackage[sans]{dsfont}
\usepackage[mathscr]{euscript}
\usepackage{mathtools}
%\usepackage{cite}
\usepackage{multido}

\graphicspath{{figs/}}

\DeclareMathOperator{\rank}{rank}      
\DeclareMathOperator{\lexmin}{lexmin}
\DeclareMathOperator{\arglexmin}{arglexmin}
\DeclareMathOperator{\argmin}{argmin}
\DeclareMathOperator{\relint}{relint}                                % relative interior
\DeclareMathOperator{\inter}{int}                                    % interior of a set
\DeclareMathOperator{\gph}{gph}                                      % graph of a mapping
\DeclareMathOperator{\epi}{epi}                                      % epigraph of a function
\DeclareMathOperator{\dom}{dom}                                      % domain of a function
\DeclareMathOperator{\bnd}{bnd}                                      % boundary of a set
\DeclareMathOperator{\aff}{affh}                                     % affine hul
\DeclareMathOperator{\co}{convh}                                     % convex hull
\DeclareMathOperator{\clco}{\overline{convh}}                        % closed convex hull

\newrgbcolor{orange}{1.00 0.65 0.00}
\newrgbcolor{mellow}{0.847 0.72 0.525}
\newrgbcolor{grey}{0.5 0.5 0.5}
\newrgbcolor{darkgrey}{0.7 0.7 0.7}
\newrgbcolor{lightgrey}{0.3 0.3 0.3}
\newrgbcolor{verylightgrey}{0.9 0.9 0.9}
\newrgbcolor{darkred}{0.5 0 0}
\newrgbcolor{darkgreen}{0 0.5 0}
\newrgbcolor{darkblue}{0 0 0.5}
\newrgbcolor{copper}{1 0.62 0.4}
\newrgbcolor{aquamarine}{0.49 1 0.83}
\newrgbcolor{pink}{1 0.4 0.6}
\newrgbcolor{lightblue}{.30 .86 .89}
\newrgbcolor{highlight}{1 1 0.9}
\newrgbcolor{periwinkle}{0.8 0.8 1}
\newrgbcolor{indigo}{0.4 0 1}
\newrgbcolor{babyblue}{0.43 1 1}
\newrgbcolor{royalblue}{0.031 0.298 0.619}
\newrgbcolor{navyblue}{0 0 0.501}
\newrgbcolor{sapphire}{0.031 0.145 0.403}
\newrgbcolor{denim}{0.082 0.376 0.741}
\newrgbcolor{cobaltblue}{0 0.278 0.67}
\newrgbcolor{ceruleanblue}{0.164 0.321 0.745}

\def\mc{\mathcal}
\def\mbf#1{\mathbf{#1}}
\def\mbb#1{\mathbb{#1}}
\def\bb{(\cdot)}
\def\bbb{(\cdot,\cdot)}
\def\bbbb{(\cdot,\cdot,\cdot)}
\def\eqbyd{:=}                                                      %equal by definition
\def\st{\mbox{ such that }}
\def\R{\mathds{R}}                                                  % real numbers
\def\Rinf{\bar{\R}}                                                 % R U +-\infty
\def\Rnm#1#2{\R^{#1\times#2}}                                         % for matrices
\def\N{\mathds{N}}                                                  % N = {1,2,...}
\def\powset#1{2^{#1}}                                               % power set
\def\proj#1#2{\mathrm{Proj}_{#1}\left(#2\right)}
\def\BackSet#1{\mathscr{B}\left(#1\right)}
\def\Bmaxmin#1{\mathscr{B}_{\small\mbox{max-min}}\left(#1\right)}
\def\Bminmax#1{\mathscr{B}_{\small\mbox{min-max}}\left(#1\right)}
\def\Smaxmin#1{\mathscr{S}_{\small\mbox{max-min}}\left(#1\right)}
\def\Sminmax#1{\mathscr{S}_{\small\mbox{min-max}}\left(#1\right)}
\def\restrict#1#2{#1 \mid #2}
\def\row#1#2{#1_{\left[#2,\cdot\right]}}                            % row of a matrix
\def\col#1#2{#1_{\left[#2\right]}}                                  % column of a vector
\def\tto{\rightrightarrows}                                         % double arrow for set-valued maps
\def\dash{\nobreakdash-\hspace{0pt}}
\def\infsup{\textrm{inf}\dash\textrm{sup} }
\def\supinf{\textrm{sup}\dash\textrm{inf} }
\def\minmax{\textrm{min}\dash\textrm{max} }
\def\maxmin{\textrm{max}\dash\textrm{min} }                       
\def\dprod#1{\langle{#1}\rangle}                                    % dot product
\def\CI#1#2{\mathcal{I}_c(#1,#2)}                                   % controllability index
\def\g#1{g_{\mbox{\tiny $#1$}}}                                     % gauge function
\def\rb#1{{\color{royalblue}{#1}}}
\def\bcite#1{\rb{\small #1}}
%

%
% Beamer setup
%
\mode<presentation>
{
  \usetheme{Boadilla}
  \setbeamercovered{dynamic}
}

\setbeamertemplate{navigation symbols}{}
\setlength{\TPHorizModule}{1cm}
\setlength{\TPVertModule}{1cm}

\usefonttheme{professionalfonts}

\usepackage{array}
\usepackage{subfigure}
\usepackage{multirow}
\usepackage{arydshln}
\usepackage{verbatim}
\usepackage{graphicx}
\usepackage{amsmath}
%\usepackage{url}
\usepackage{hyperref}
\usepackage{breakurl}
%\usepackage{pdfpages}

\makeatletter
\newcommand{\verbatimfont}[1]{\def\verbatim@font{#1}}%
\makeatother

\setbeamertemplate{navigation symbols}{}
\setlength{\TPHorizModule}{1cm}
\setlength{\TPVertModule}{1cm}


\usefonttheme{professionalfonts}
\title[Calling MINLP solvers from Julia]
{Calling nonlinear and MINLP solvers from Julia\\
\vspace{5pt}
\small Interfaces, formats, expression trees and AD tools
\vspace{-10pt}}

\author{Tony Kelman}

\institute[UC Berkeley]
{
  University of California, Berkeley \\
  
  \vspace{5pt}
  {\fontfamily{pcr}\selectfont kelman@berkeley.edu} \\

  \vspace{5pt}
  \url{https://github.com/tkelman}
}
\date[ISMP, July 16 2015]{\includegraphics[width=85pt]{fig/juliaopt_logo.eps}
\\ISMP, Pittburgh PA\\July 16, 2015}

\subject{}


\begin{document}
\verbatimfont{\fontfamily{pcr}\selectfont}
%\setbeamertemplate{blocks}[rounded][shadow=true]

\setbeamertemplate{footline}{}
\begin{frame}
  \titlepage
\end{frame}


%
%===========================================================================================
%
\setcounter{framenumber}{0}
\setbeamertemplate{footline}[infolines theme]



\begin{frame}[fragile]\frametitle{In the spirit of reproducible research}

You can get to solving a (small) MINLP before the end of this talk
\footnote{On Mac or Windows, where we download precompiled solver binaries \\
~~~~ On Linux, has to compile solvers from source which takes a few minutes}

  \begin{enumerate}
    \item Install a binary of the latest Julia release at \\
      \url{http://julialang.org/downloads}
    \item Open a Julia session and run: \\
      \texttt{Pkg.add("JuMP") \\
      Pkg.add("CoinOptServices") \\
      using JuMP, CoinOptServices, Compat \\
      include(Pkg.dir("JuMP","test","solvers.jl")); \\
      include(Pkg.dir("JuMP","test","nonlinear.jl"));}
  \end{enumerate}

Will come back to an example problem later

\end{frame}
%
%--------------------------------------------------------------
%



\begin{frame}[fragile]\frametitle{Talk outline}
\begin{itemize}
  \item MINLP problem statement, standard form
  \item Expression trees
  \item Automatic differentiation
  \item Solver interfaces
  \begin{itemize}
    \item Data formats, COIN OSiL and AMPL nl
    \item APIs, connecting to high level languages
  \end{itemize}
  \item Julia implementations
  \begin{itemize}
    \item JuMP and MathProgBase
    \item CoinOptServices and AmplNLWriter
  \end{itemize}
  \item Example problems
  \item Performance evaluation
\end{itemize}

\end{frame}
%
%--------------------------------------------------------------
%



\begin{frame}[fragile]\frametitle{Mixed integer nonlinear programming}

\begin{block}{\center MINLP problem statement}
\vspace{-15pt}
\begin{align*}
\min_x ~ & f(x) \\
\text{s.t.} ~ & g(x) \leq 0 \\
& h(x) = 0 \\
& x \in \mathbb{R}^n \\
& x_i \in \mathbb{Z} ~ \forall ~ i \in \mathcal{I}
\end{align*}
\end{block}

\begin{itemize}
  \item For this talk $f(x)$, $g(x)$ may be non-convex, $h(x)$ may be nonlinear
  \item Assume that $f(x), g(x), h(x)$ are algebraic expressions
  \item What is the ``standard form" to encode this problem?
\end{itemize}

\end{frame}
%
%--------------------------------------------------------------
%




\begin{frame}[fragile]\frametitle{Expression trees}
\vspace{-15pt}
\begin{minipage}{0.55\textwidth}
\begin{figure}
\includegraphics[width=135pt]{fig/expr_tree.eps}
\end{figure}

Julia has a first-class expression type and quote operator,
it already does exactly what we want here \\

Real macros and metaprogramming ({\fontfamily{pcr}\selectfont @addNLConstraint} in JuMP) take much of the work out of this

\end{minipage}
\begin{minipage}{0.35\textwidth}
{\fontsize{6.8pt}{0} \selectfont
\begin{verbatim}
julia> dump(:( 2*3/(2-1)+5*(4-1) ), 6)
Expr
  head: Symbol call
  args: Array(Any,(3,))
    1: Symbol +
    2: Expr
      head: Symbol call
      args: Array(Any,(3,))
        1: Symbol /
        2: Expr
          head: Symbol call
          args: Array(Any,(3,))
            1: Symbol *
            2: Int64 2
            3: Int64 3
          typ: Any
        3: Expr
          head: Symbol call
          args: Array(Any,(3,))
            1: Symbol -
            2: Int64 2
            3: Int64 1
          typ: Any
      typ: Any
    3: Expr
      head: Symbol call
      args: Array(Any,(3,))
        1: Symbol *
        2: Int64 5
        3: Expr
          head: Symbol call
          args: Array(Any,(3,))
            1: Symbol -
            2: Int64 4
            3: Int64 1
          typ: Any
      typ: Any
  typ: Any
\end{verbatim}
}
\end{minipage}
\end{frame}
%
%--------------------------------------------------------------
%


\begin{frame}[fragile]\frametitle{Automatic (algorithmic) differentiation}

\begin{itemize}
\item Large body of literature and implementations (ADOL-C, CppAD, AMPL ASL, ReverseDiffSparse.jl, etc), see \url{http://www.mit.edu/~mlubin/informs2014-nlp.pdf} for more details
\item Given expression trees of objective and constraint functions $f(x), g(x), h(x)$, AD provides efficient calculation method for sparse constraint Jacobian and Lagrangian Hessian
\item For general nonlinear problems that don't fit in a special form, you should be using AD unless you have a good reason not to
\end{itemize}

\end{frame}
%
%--------------------------------------------------------------
%



\begin{frame}[fragile]\frametitle{Data formats for representing expression trees}

\begin{itemize}
\item State of the art in practice: AMPL .nl format, see ``Hooking Your Solver to AMPL" and ``Writing .nl Files" technical reports by David M. Gay
\item High level human-readable .mod file converted to low level expression tree .nl format by proprietary AMPL modeling layer
\item Solvers link to open-source AMPL solver library (ASL) for function, Jacobian, Hessian evaluation from .nl file
\item Now available in Julia thanks to Jack Dunn's AmplNLWriter.jl package, JuMP can create .nl files
\end{itemize}

\end{frame}
%
%--------------------------------------------------------------
%


\begin{frame}[fragile]\frametitle{Creating a .nl file from JuMP}
Copy-paste from \url{http://bit.do/ismp-minlp}
{\fontsize{8.8pt}{0} \selectfont
\begin{verbatim}
Pkg.add("AmplNLWriter")
using JuMP, AmplNLWriter
# Solve test problem 1 (Synthesis of processing system) in
# M. Duran & I.E. Grossmann, "An outer approximation algorithm for
# a class of mixed integer nonlinear programs", Mathematical
# Programming 36, pp. 307-339, 1986.
m = Model(solver = BonminNLSolver())
x_U = [2,2,1]
@defVar(m, x_U[i] >= x[i=1:3] >= 0)
@defVar(m, y[4:6], Bin)
@setNLObjective(m, Min, 10 + 10*x[1] - 7*x[3] + 5*y[4] + 6*y[5] +
    8*y[6] - 18*log(x[2]+1) - 19.2*log(x[1]-x[2]+1))
@addNLConstraints(m, begin
    0.8*log(x[2] + 1) + 0.96*log(x[1] - x[2] + 1) - 0.8*x[3] >= 0
    log(x[2] + 1) + 1.2*log(x[1] - x[2] + 1) - x[3] - 2*y[6] >= -2
    x[2] - x[1] <= 0
    x[2] - 2*y[4] <= 0
    x[1] - x[2] - 2*y[5] <= 0
    y[4] + y[5] <= 1
end)
status = solve(m)
\end{verbatim}
}
\end{frame}
%
%--------------------------------------------------------------
%


\begin{frame}[fragile]\frametitle{The resulting .nl file}
\begin{minipage}{0.35\textwidth}
{\fontsize{6.8pt}{0} \selectfont
\begin{verbatim}
g3 1 1 0
 6 6 1 6 0 0
 2 1
 0 0
 2 2 2
 0 0 0 0
 3 0 0 0 0
 16 6
 0 0
 0 0 0 0 0
C0
o0
o2
n0.8
o43
o0
v1
n1
o2
n0.96
o43
o0
o1
v0
v1
n1
C1
o0
o43
o0
v1
n1
o2
n1.2
o43
o0
o1
v0
v1
n1
C2
n0
C3
n0
\end{verbatim}
}
\end{minipage}
\begin{minipage}{0.35\textwidth}
{\fontsize{6.8pt}{0} \selectfont
\begin{verbatim}
C4
n0
C5
n0
O0 0
o0
o1
o16
o2
n18
o43
o0
v1
n1
o2
n19.2
o43
o0
o1
v0
v1
n1
n10
d6
0 0
1 0
2 0
3 0
4 0
5 0
x6
0 0.0
1 0.0
2 0.0
3 0.0
4 0.0
5 0.0
r
2 0.0
2 -2.0
1 0.0
1 0.0
1 0.0
1 1.0
\end{verbatim}
}
\end{minipage}
\begin{minipage}{0.25\textwidth}
{\fontsize{6.8pt}{0} \selectfont
\begin{verbatim}
b
0 0.0 2.0
0 0.0 2.0
0 0.0 1.0
0 0.0 1.0
0 0.0 1.0
0 0.0 1.0
k5
4
9
11
13
15
J0 3
0 0.0
1 0.0
2 -0.8
J1 4
0 0.0
1 0.0
2 -1.0
5 -2.0
J2 2
0 -1.0
1 1.0
J3 2
1 1.0
3 -2.0
J4 3
0 1.0
1 -1.0
4 -2.0
J5 2
3 1.0
4 1.0
G0 6
0 10.0
1 0.0
2 -7.0
3 5.0
4 6.0
5 8.0
\end{verbatim}
}
\end{minipage}
\end{frame}
%
%--------------------------------------------------------------
%


\begin{frame}\frametitle{COIN-OR solvers and Optimization Services}
\begin{tabular}{m{3.0in} m{2in}}
\begin{itemize}
\item Comprehensive stack of cutting-edge \\
open source solvers and libraries
\item CLP: continuous linear programming
\item CBC: mixed-integer linear programming
\item Ipopt: continuous nonlinear programming
\item Bonmin: evaluation based mixed-integer nonlinear programming
\item Couenne: expression tree based mixed-integer nonlinear programming
\item CppAD: automatic differentiation
\item Optimization Services: standardized interchange formats,
remote solution protocols, solver-independent interfaces
\end{itemize}
&
\includegraphics[width=1.1in]{./fig/COIN-OR.eps}
\vspace{100pt}
\end{tabular}
\end{frame}
%
%--------------------------------------------------------------
%



\begin{frame}[fragile]\frametitle{Optimization Services OSiL format - 1}
\begin{itemize}
\item XML based, human readable
\item \texttt{OSSolverService} driver uses CppAD for Jacobian, Hessian
\item Same Julia code, except for: \\
{\footnotesize
\texttt{using CoinOptServices \\
m = Model(solver = OsilBonminSolver())}}
\end{itemize}
{\fontsize{6pt}{0} \selectfont
\begin{verbatim}
<?xml version="1.0" encoding="utf-8"?>
<osil xmlns="os.optimizationservices.org"
    xmlns:xsi="http://www.w3.org/2001/XMLSchema-instance"
    xsi:schemaLocation="os.optimizationservices.org
    http://www.optimizationservices.org/schemas/2.0/OSiL.xsd">
  <instanceHeader>
    <description>generated by CoinOptServices.jl</description>
  </instanceHeader>
  <instanceData>
    <variables numberOfVariables="6">
      <var lb="0.0" ub="2.0" type="C"/>
      <var lb="0.0" ub="2.0" type="C"/>
      <var lb="0.0" ub="1.0" type="C"/>
      <var lb="0.0" ub="1.0" type="B"/>
      <var lb="0.0" ub="1.0" type="B"/>
      <var lb="0.0" ub="1.0" type="B"/>
    </variables>
    <objectives numberOfObjectives="1">
      <obj maxOrMin="min" numberOfObjCoef="0"/>
    </objectives>
    <constraints numberOfConstraints="6">
      <con lb="0.0"/>
      <con lb="0.0"/>
      <con ub="0.0"/>
      <con ub="0.0"/>
      <con ub="0.0"/>
      <con ub="0.0"/>
    </constraints>
\end{verbatim}
}
\end{frame}
%
%--------------------------------------------------------------
%


\begin{frame}[fragile]\frametitle{Optimization Services OSiL format - 2}
\begin{minipage}{0.54\textwidth}
{\fontsize{6pt}{0} \selectfont
\begin{verbatim}
<nonlinearExpressions numberOfNonlinearExpressions="7">
  <nl idx="-1">
    <minus>
      <minus>
        <sum>
          <minus>
            <plus>
              <number value="10"/>
              <variable idx="0" coef="10"/>
            </plus>
            <variable idx="2" coef="7"/>
          </minus>
          <variable idx="3" coef="5"/>
          <variable idx="4" coef="6"/>
          <variable idx="5" coef="8"/>
        </sum>
        <times>
          <number value="18"/>
          <ln>
            <plus>
              <variable idx="1"/>
              <number value="1"/>
            </plus>
          </ln>
        </times>
      </minus>
      <times>
        <number value="19.2"/>
        <ln>
          <plus>
            <minus>
              <variable idx="0"/>
              <variable idx="1"/>
            </minus>
            <number value="1"/>
          </plus>
        </ln>
      </times>
    </minus>
  </nl>
\end{verbatim}
}
\end{minipage}
\begin{minipage}{0.42\textwidth}
{\fontsize{6pt}{0} \selectfont
\begin{verbatim}
      <nl idx="0">
        <minus>
          <minus>
            <plus>
              <times>
                <number value="0.8"/>
                <ln>
                  <plus>
                    <variable idx="1"/>
                    <number value="1"/>
                  </plus>
                </ln>
              </times>
              <times>
                <number value="0.96"/>
                <ln>
                  <plus>
                    <minus>
                      <variable idx="0"/>
                      <variable idx="1"/>
                    </minus>
                    <number value="1"/>
                  </plus>
                </ln>
              </times>
            </plus>
            <variable idx="2" coef="0.8"/>
          </minus>
          <number value="0"/>
        </minus>
      </nl>
\end{verbatim}
}
\end{minipage}
\end{frame}
%
%--------------------------------------------------------------
%


\begin{frame}[fragile]\frametitle{Optimization Services OSiL format - 3}
\begin{minipage}{0.54\textwidth}
{\fontsize{6pt}{0} \selectfont
\begin{verbatim}
  <nl idx="1">
    <minus>
      <minus>
        <minus>
          <plus>
            <ln>
              <plus>
                <variable idx="1"/>
                <number value="1"/>
              </plus>
            </ln>
            <times>
              <number value="1.2"/>
              <ln>
                <plus>
                  <minus>
                    <variable idx="0"/>
                    <variable idx="1"/>
                  </minus>
                  <number value="1"/>
                </plus>
              </ln>
            </times>
          </plus>
          <variable idx="2"/>
        </minus>
        <variable idx="5" coef="2"/>
      </minus>
      <number value="-2"/>
    </minus>
  </nl>
  <nl idx="2">
    <minus>
      <minus>
        <variable idx="1"/>
        <variable idx="0"/>
      </minus>
      <number value="0"/>
    </minus>
  </nl>
\end{verbatim}
}
\end{minipage}
\begin{minipage}{0.42\textwidth}
{\fontsize{6pt}{0} \selectfont
\begin{verbatim}
      <nl idx="3">
        <minus>
          <minus>
            <variable idx="1"/>
            <variable idx="3" coef="2"/>
          </minus>
          <number value="0"/>
        </minus>
      </nl>
      <nl idx="4">
        <minus>
          <minus>
            <minus>
              <variable idx="0"/>
              <variable idx="1"/>
            </minus>
            <variable idx="4" coef="2"/>
          </minus>
          <number value="0"/>
        </minus>
      </nl>
      <nl idx="5">
        <minus>
          <plus>
            <variable idx="3"/>
            <variable idx="4"/>
          </plus>
          <number value="1"/>
        </minus>
      </nl>
    </nonlinearExpressions>
  </instanceData>
</osil>
\end{verbatim}
}
\end{minipage}
\end{frame}
%
%--------------------------------------------------------------
%



\begin{frame}[fragile]\frametitle{Solver APIs and connecting to high level languages}
\begin{itemize}
\item Convenient for users to work in high level languages for model formulation, data processing, function evaluation (if fast enough)
\item Modern solvers in \texttt{C++} are hard to use from other languages unless they have a dedicated \texttt{C} API
\begin{itemize}
\item e.g. Bonmin doesn't need expression trees, but only has a \texttt{C++} API
\item Julia solver interfaces very easy when there is a \texttt{C} API: \texttt{ccall((function\_name, library\_name), return\_type, (input\_types,), inputs)}
\item Direct Julia to \texttt{C++} interface being worked on, but still unstable
\end{itemize}
\onslide<2->{\item Julia is both fast and high level, convenient for developers too
\begin{itemize}
\item No need to prototype in Matlab/Python then rewrite in \texttt{C}/\texttt{C++} for performance
\item Pure-Julia AD in ReverseDiffSparse.jl is fast enough that it's not the bottleneck
\item Can stay in memory, avoid .nl or .osil files, allow efficient re-solves
\item Solvers entirely in Julia? Very little code to connect to MathProgBase, JuMP, Convex.jl
\end{itemize}}
\end{itemize}
\end{frame}
%
%--------------------------------------------------------------
%


\begin{frame}[fragile]\frametitle{MathProgBase extensibility and new solvers}
\begin{center}
\includegraphics[width=150pt]{fig/mathprogbase.eps}
\end{center}
\vspace{-15pt}
\begin{itemize}
\item Unified Julia API for all solver packages and modeling tools to talk to
\item Solvers call \texttt{eval\_jac\_g} to evaluate constraint Jacobian
\item AmplNLWriter and CoinOptServices call \texttt{obj\_expr}, \texttt{constr\_expr} to get Julia expression trees for objective and constraint functions
\item More solvers, more modeling environments (AmplNLReader.jl), extensions of JuMP and Convex.jl
\end{itemize}
\end{frame}
%
%--------------------------------------------------------------
%



\begin{frame}[fragile]\frametitle{Larger example problem}
\begin{itemize}
\item Goddard rocket control problem from COPS3 (continuous NLP)
\url{http://bit.do/rocket-jump}
\item Comparing AD function evaluation performance between
\begin{enumerate}
\item Pure-Julia ReverseDiffSparse.jl: {\footnotesize \texttt{solver=IpoptSolver(print\_timing\_statistics="yes")}}
\item ASL via nl: {\footnotesize \texttt{solver=IpoptNLSolver()}}
\item CppAD via osil: {\footnotesize \texttt{solver=OsilSolver(OSOption("print\_timing\_statistics", "yes"))}}
\end{enumerate}
\end{itemize}
\end{frame}
%
%--------------------------------------------------------------
%


\begin{frame}[fragile]\frametitle{Pure-Julia AD}
{\footnotesize \texttt{solver=IpoptSolver(print\_timing\_statistics="yes")}}
{\fontsize{6pt}{0} \selectfont
\begin{verbatim}
Total CPU secs in IPOPT (w/o function evaluations)   =      0.719
Total CPU secs in NLP function evaluations           =      0.412
                                                                                                       ~
                                                                                                       ~
Timing Statistics:
                                                                                                       ~
OverallAlgorithm....................:      1.131 (sys:      0.000 wall:      1.131)
 PrintProblemStatistics.............:      0.021 (sys:      0.000 wall:      0.021)
 InitializeIterates.................:      0.101 (sys:      0.000 wall:      0.101)
 UpdateHessian......................:      0.317 (sys:      0.000 wall:      0.317)
 OutputIteration....................:      0.138 (sys:      0.000 wall:      0.138)
 UpdateBarrierParameter.............:      0.003 (sys:      0.000 wall:      0.003)
 ComputeSearchDirection.............:      0.418 (sys:      0.000 wall:      0.418)
 ComputeAcceptableTrialPoint........:      0.040 (sys:      0.000 wall:      0.040)
 AcceptTrialPoint...................:      0.000 (sys:      0.000 wall:      0.000)
 CheckConvergence...................:      0.071 (sys:      0.000 wall:      0.071)
PDSystemSolverTotal.................:      0.417 (sys:      0.000 wall:      0.417)
 PDSystemSolverSolveOnce............:      0.396 (sys:      0.000 wall:      0.396)
 ComputeResiduals...................:      0.018 (sys:      0.000 wall:      0.018)
 StdAugSystemSolverMultiSolve.......:      0.445 (sys:      0.000 wall:      0.445)
 LinearSystemScaling................:      0.000 (sys:      0.000 wall:      0.000)
 LinearSystemSymbolicFactorization..:      0.044 (sys:      0.000 wall:      0.044)
 LinearSystemFactorization..........:      0.000 (sys:      0.000 wall:      0.000)
 LinearSystemBackSolve..............:      0.058 (sys:      0.000 wall:      0.058)
 LinearSystemStructureConverter.....:      0.000 (sys:      0.000 wall:      0.000)
  LinearSystemStructureConverterInit:      0.000 (sys:      0.000 wall:      0.000)
Function Evaluations................:      0.412 (sys:      0.000 wall:      0.412)
 Objective function.................:      0.000 (sys:      0.000 wall:      0.000)
 Objective function gradient........:      0.000 (sys:      0.000 wall:      0.000)
 Equality constraints...............:      0.024 (sys:      0.000 wall:      0.024)
 Inequality constraints.............:      0.000 (sys:      0.000 wall:      0.000)
 Equality constraint Jacobian.......:      0.072 (sys:      0.000 wall:      0.072)
 Inequality constraint Jacobian.....:      0.000 (sys:      0.000 wall:      0.000)
 Lagrangian Hessian.................:      0.316 (sys:      0.000 wall:      0.316)
\end{verbatim}
}
\end{frame}
%
%--------------------------------------------------------------
%



\begin{frame}[fragile]\frametitle{ASL via nl}
{\footnotesize \texttt{solver=IpoptNLSolver()}}
{\fontsize{6pt}{0} \selectfont
\begin{verbatim}
Total CPU secs in IPOPT (w/o function evaluations)   =      0.732
Total CPU secs in NLP function evaluations           =      0.244
                                                                                                       ~
                                                                                                       ~
Timing Statistics:
                                                                                                       ~
OverallAlgorithm....................:      0.976 (sys:      0.000 wall:      0.976)
 PrintProblemStatistics.............:      0.021 (sys:      0.000 wall:      0.021)
 InitializeIterates.................:      0.055 (sys:      0.000 wall:      0.055)
 UpdateHessian......................:      0.174 (sys:      0.000 wall:      0.175)
 OutputIteration....................:      0.139 (sys:      0.000 wall:      0.139)
 UpdateBarrierParameter.............:      0.001 (sys:      0.000 wall:      0.001)
 ComputeSearchDirection.............:      0.470 (sys:      0.000 wall:      0.470)
 ComputeAcceptableTrialPoint........:      0.057 (sys:      0.000 wall:      0.057)
 AcceptTrialPoint...................:      0.000 (sys:      0.000 wall:      0.000)
 CheckConvergence...................:      0.032 (sys:      0.000 wall:      0.032)
PDSystemSolverTotal.................:      0.470 (sys:      0.000 wall:      0.470)
 PDSystemSolverSolveOnce............:      0.454 (sys:      0.000 wall:      0.454)
 ComputeResiduals...................:      0.011 (sys:      0.000 wall:      0.011)
 StdAugSystemSolverMultiSolve.......:      0.483 (sys:      0.000 wall:      0.483)
 LinearSystemScaling................:      0.000 (sys:      0.000 wall:      0.000)
 LinearSystemSymbolicFactorization..:      0.026 (sys:      0.000 wall:      0.026)
 LinearSystemFactorization..........:      0.000 (sys:      0.000 wall:      0.000)
 LinearSystemBackSolve..............:      0.058 (sys:      0.000 wall:      0.058)
 LinearSystemStructureConverter.....:      0.000 (sys:      0.000 wall:      0.000)
  LinearSystemStructureConverterInit:      0.000 (sys:      0.000 wall:      0.000)
Function Evaluations................:      0.244 (sys:      0.000 wall:      0.244)
 Objective function.................:      0.000 (sys:      0.000 wall:      0.000)
 Objective function gradient........:      0.003 (sys:      0.000 wall:      0.003)
 Equality constraints...............:      0.039 (sys:      0.000 wall:      0.039)
 Inequality constraints.............:      0.000 (sys:      0.000 wall:      0.000)
 Equality constraint Jacobian.......:      0.028 (sys:      0.000 wall:      0.028)
 Inequality constraint Jacobian.....:      0.000 (sys:      0.000 wall:      0.000)
 Lagrangian Hessian.................:      0.174 (sys:      0.000 wall:      0.174)
\end{verbatim}
}
\end{frame}
%
%--------------------------------------------------------------
%



\begin{frame}[fragile]\frametitle{CppAD via osil}
{\footnotesize \texttt{solver=OsilSolver(OSOption("print\_timing\_statistics", "yes"))}}
\begin{center}
\includegraphics[width=120pt]{fig/bug.eps}
\end{center}
\vspace{-15pt}
\begin{itemize}
\item Failed to converge, initial conditions not set right by \texttt{OSSolverService}
\item Function evaluation time several hundred times slower per iteration
\item Interesting disagreeement on ``Number of nonzeros in Lagrangian Hessian" for same problem
\begin{enumerate}
\item ReverseDiffSparse.jl: 48800
\item ASL: 11214
\item OSSolverService: 8811
\end{enumerate}
\end{itemize}
\end{frame}
%
%--------------------------------------------------------------
%



\begin{frame}[fragile]\frametitle{Conclusions}
\begin{itemize}
  \item There are some bugs in Optimization Services, need to report them along with .nl and .osil test files, and spend some time profiling in \texttt{C++}
  \item Only a few hundred lines of Julia code to implement expression tree to .osil (or .nl) conversion
  \item For now use CoinOptServices to install Bonmin and Couenne solver binaries, AmplNLWriter for MINLP's, and either AmplNLWriter or in-memory interfaces for continuous NLP's
  \item Want advice on writing Julia bindings to your solver, or how to start prototyping algorithms in Julia? {\footnotesize \fontfamily{pcr}\selectfont julia-opt@googlegroups.com}
\end{itemize}

\end{frame}
%
%--------------------------------------------------------------
%



\end{document}
